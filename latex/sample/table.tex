\documentclass{article}
\usepackage{multirow}
\begin{document}
LaTeX table example\\
\verb= http:\\www.chinatex.org=\\
\begin{table}[!hbp]
\begin{tabular}{|c|c|c|c|c|}
\hline
\hline
lable 1-1 & label 1-2 & label 1-3 & label 1 -4 & label 1-5 \\
\hline
label 2-1 & label 2-2 & label 3-3 & label 4-4 & label 5-5 \\
\hline
\multirow{2}{*}{Multi-Row} & \multicolumn{2}{|c|}{Multi-Column} & \multicolumn{2}{|c|}{\multirow{2}{*}{Multi-Row and Col}} \\
\cline{2-3}
& column-1 & column-2 & \multicolumn{2}{|c|}{}\\
\hline
\end{tabular}
\caption{My first table}
\end{table}
\end{document}

\documentclass{article}%开始文档
\usepackage{multirow}%使用多栏宏包
\begin{document}%开始文档
LaTeX table example\\
\verb= http:\\www.chinatex.org =\\
\begin{table}[!hbp]%开始表格
%其中参数[!hbp] 的意思是:
%!表示尽可能的尝试 h(here) 当前位置显示表格,
%如果实在不行显示在 b(bottom) 底部,
\begin{tabular}{|c|c|c|c|c|}%开始绘制表格
%{|c|c|c|c|c|} 表示会有5列, 每个的方式未居中(c),
%也可以改成靠左(l)和靠右(r) 其中 | 表示绘制列线
\hline %绘制一条水平的线
\hline %再绘制一条水平的线
lable 1-1 & label 1-2 & label 1-3 & label 1 -4 & label 1-5
%这事表格的第一行, 其中5个元素, 用 &隔开.
\hline
label 2-1 & label 2-2 & label 3-3 & label 4-4 & label 5-5 \\
%这事表格的第二行, 其中5个元素, 用 & 隔开.
\hline
%下面这一段有点复杂,参加后面的解释,可以自己修改慢慢体会.
\multirow{2}{*}{Multi-Row} & \multicolumn{2}{|c|}{Multi-Column} & \multicolumn{2}{|c|}{\multirow{2}{*}{Multi-Row and Col}} \\
%上面开始两行合并, 然后又是正常的两列合并, 接下来是两行两列合并
\cline{2-3} %绘制第2列和第3列的横线
& column-1 & column-2 & \multicolumn{2}{|c|}{}\\
%补偿上面的两列合并的那一行
\hline
\end{tabular}
\caption{My first table} %表格的名称
\end{table}
\end{document}
