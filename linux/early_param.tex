\documentclass[b5paper,9pt,twoside,openany]{article}

\usepackage{fontspec,xunicode,xltxtra}
\usepackage{listings}
\usepackage{color}
\usepackage{paralist}
\usepackage{qtree}
\usepackage{dirtree}
\usepackage{pstricks,pst-node}
\usepackage{tabularx}
\usepackage[a5paper]{geometry}
\usepackage[colorlinks=true,linkcolor=blue]{hyperref}


\XeTeXlinebreaklocale ``zh''
\XeTeXlinebreakskip = 0pt plus 1pt minus 0.1pt
\newfontfamily\song{SimSun}
\newfontfamily\hei{SimHei}
%\newfontfamily\kai{KaiTi}
%\newfontfamily\fsong{FangSong}
%\newfontfamily\nsong{NSimSun}
\newfontfamily\mshei{Microsoft YaHei}
\setmainfont{SimSun}

\begin{document}
\title{early-param分析}
\author{lenxyang}
\maketitle
\section{early-param}
early-param是Linux早期初始化的框架,它通过字符串及函数指针建立一个初始化表。在进入保护模式之前,Linux将实模式下通过BIOS提供的中断功能获取很多参数,在进入保护模式之后Linux将通过之前的配置项对这些参数进行解析而后根据参数进行初始化。

early-param负责初始化的功能包含:

\begin{tabular}{|c|c|c|}
\hline
早期的printk | early-printk | arch/x86/kernel/early-printk.c \\
\hline
e820内存 | initmem  | arch/x86/kernel/e820.c \\
\end{tabular}
\section{early-param配置方式}

\section{early-param框架}
\end{document}
