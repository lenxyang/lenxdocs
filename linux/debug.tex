\documentclass[b5paper,9pt,twoside,openany]{article}

\usepackage{fontspec,xunicode,xltxtra}
\usepackage{listings}
\usepackage{color}
\usepackage{paralist}
\usepackage{qtree}
\usepackage{dirtree}
\usepackage{pstricks,pst-node}
\usepackage{tabularx}
\usepackage[a5paper]{geometry}
\usepackage[colorlinks=true,linkcolor=blue]{hyperref}


\XeTeXlinebreaklocale ``zh''
\XeTeXlinebreakskip = 0pt plus 1pt minus 0.1pt
\newfontfamily\song{SimSun}
\newfontfamily\hei{SimHei}
%\newfontfamily\kai{KaiTi}
%\newfontfamily\fsong{FangSong}
%\newfontfamily\nsong{NSimSun}
\newfontfamily\mshei{Microsoft YaHei}
\setmainfont{SimSun}

\begin{document}
\title{使用gdb+qemu调试Linux}
\author{lenxyang}
\maketitle
\section{编译内核}
编译内核是个既简单又复杂的任务,最简单的只要执行
\begin{lstlisting}
make menuconfig
make
make install
\end{lstlisting}
编译内核就算完成了,但这种内核肯定是不能用于跟踪调试的。它包含的内容太多,、
\begin{itemize}
\item SMP架构支持
\item x86 PAE, PSE的支持
\item APIC
\item 各类driver
\end{itemize}
有太多的东西影响实现,让内核看起来非常复杂,没一部分都要认真的去读Intel的手册,而为了阅读那么由简入繁,某些硬件细节还是应该尽量去掉。
\subsection{去除不必要的功能}
\begin{itemize}
\item Processor type and features
\begin{itemize}
\item 去除Symmetric multi-processing support
\item 选择"High Memory Support (4GB)"
\item 去掉Allocate 3rd-level pagetables from highmem
\item Generic Driver Options
\end{itemize}
\item Device Drivers
\begin{itemize}
\item Multimedia support
\item Sound card support
\item HID Devices
\item Sony MemoryStick card support
\end{itemize}
\item{File systems}
\begin{itemize}
\item The Extended 4 (ext4) filesystem
\item Network File Systems
\end{itemize}
\end{itemize}

\subsection{添加完整的调试信息}
完整的调试信息能够帮助我们方便的跟踪内核的执行路径,调用栈,观察变量和内存的变化等等。这对于初学内核的人都有莫大的帮助。
\begin{itemize}
\item 选中Kernel Hacking当中的“Compile the kernel with debug info”
这一点比较容易理解,在编译完成之后会生成一个带有调试信息的vmlinux文件,此文件当中包含对应的符号信息及源代码信息。gdb可以将此文件作为符号文件加载以进行源代码级调试,这对于跟踪代码来说是必不可少的。
\item 在General setup当中去掉 “Optimize for size”并将Makefile当中的“KBUILD\_CFLAGS   += -O2”的-O2改为-O0
带有调试信息之后,用户可以进行源代码级调试。但如果启用O2调试选项,在gdb跟踪代码时,当前代码就会跳来跳去,因为编译器调整了代码的顺序;此外一些变量被优化掉了,没有办法直接查看它的值,这对跟踪代码来说都不方便。通过将O2改为-O能够禁用掉大多数优化策略,已经够用了,根本不用优化将无法通过编译,因此可以也不得不保留一部分优化。
实际上GCC提供了很多优化选项,而-O3, -Os, -O2, -O1和-O0不过是包含了这些具体优化选项的自己,如果希望了解具体的优化选项可以通过man gcc查阅。
\item 选中Kernel Hacking当中的“Compile the kernel with frame pointers”
如果不包含此选项,GCC将使用-fomit-frame-pointer进行编译,gdb将无法打出完整的backtrace信息,给跟踪代码带来不便。

\item 选中Kernel Hacking当中的“Allow gcc to uninline functions marked 'inline'”
\end{itemize}

\subsection{可能出现的错误}
\begin{itemize}
\item 去掉“Device Drivers”->“Character devices”->“ACP Modem (Mwave) support”
在改掉-O2为-O0之后,mware会编译出问题(2.6.32.60),它是不是必要的去掉它就好。
\item arch/x86/mm/memtest.c:50: undefined reference to `\_\_udivdi3'
\item drivers/serial/serial\_core.c:759: undefined reference to `tty\_port\_users'
\item mm/slab.c:343: undefined reference to `\_\_bad\_size'
在 mm/Makefile当中增加一行
\begin{lstlisting}
CFLAGS_slab.o = -O2
\end{lstlisting}
\end{itemize}
\section{配置磁盘镜像}
\begin{lstlisting}[language=bash]
# 创建一个500M的磁盘
 bximage
 Do you want to create a floppy disk image or a hard disk image?
 Please type hd or fd. [hd] 

 What kind of image should I create?
 Please type flat, sparse or growing. [flat]

 Enter the hard disk size in megabytes, between 1 and 129023
 [10] 500

 I will create a 'flat' hard disk image with
  cyl=1015
  heads=16
  sectors per track=63
  total sectors=1023120
  total size=499.57 megabytes
\end{lstlisting}

在完成磁盘的创建以后,还需要对磁盘创建分区并格式化。Linux有一个非常方便的工具mtools可以完成功能,它提供了一组命令不需要将磁盘镜像映射成设备就可以对其操作。默认情况下linux是包含这个工具包的,可以通过命令sudo apt-get install mtools来安装。
\begin{lstlisting}[language=bash]
# mpartition -I C:
$ mpartition -cpv -t 1015 -h 16 -s 63 -b 63 C:
\end{lstlisting}

\subsection{加载磁盘镜像}
在完成磁盘的创建以后,需要对磁盘镜像以块设备的方式进行操作。
\begin{lstlisting}[language=bash]
$ sudo losetup /dev/loop0 c.img
$ ls -l /dev/loop0
brw-rw---- 1 root disk 7, 0 Feb  4 14:56 /dev/loop0 # 7, 0 是主,次设备号
# 为磁盘的分区建立设备映射
$ sudo kpartx -v -a /dev/loop0
# 将分区映射到设备 /dev/loop1上
$ sudo losetup /dev/loop1 /dev/mapper/loop0p1
# 格式化磁盘
# 注意此处应设用ext3而非ext2, 否则linux无法安装磁盘系统
$ sudo mkfs.ext3 /dev/loop1
\end{lstlisting}

\subsection{安装grub2}

\begin{lstlisting}[language=bash]
$ sudo grub-install --no-floppy \
       --root-directory=/mnt \
       --modules="fat ext2 part_msdos"  /dev/loop0
\end{lstlisting}
将c.img作为启动盘启动qemu,此时已经可以看到GRUB2的提示符号了。有一点需要注意:前面提过虚拟镜像采用ext3文件系统,而此处加载的确实ext2 modules,这不会引起什么问题。系统可以将ext3的文件系统当作ext2来加载而不会出现问题,此外GRUB2也没有ext3这个模块。
\begin{lstlisting}
qemu-system-i386 -hda c.img
\end{lstlisting}

\subsection{安装Linux}
创建文件/mnt/boot/grub/grub.cfg
\begin{lstlisting}[language=bash]

menuentry "My Linux Kernel" {
  set root='(hd0,msdos1)'
  linux /vmlinuz ro
  initrd /initrd.img
  insmod gzio
  insmod part_msdos
  insmod ext2
}
\end{lstlisting}

将生成的Linux的两个文件vmlinux和initrd.img拷贝过去。这两个文件在make install的时候会被拷贝到本地/boot目录下,从哪里copy到/mnt/下即可。

\subsection{编译busybox}
在编译之前修改编译选线
\begin{itemize}
\item 选中 "Build BusyBox as a static binary (no shared libs)"
\item 去除 "Networking Utilities"-> "inetd"
\end{itemize}
\section{创建系统目录}
\begin{lstlisting}[language=bash]
# 安装文件系统
# 将busybox生成的文件coyp /mnt下
# 创建配置文件
sudo mkdir /mnt/etc
sudo mkdir /mnt/dev/
sudo mkdir /mnt/etc/rc.d
cat > /tmp/tmp.a
#!/bin/sh
mount -t proc non /proc
mount -t sysfs non /sys
/bin/sh
mv cp /tmp/tmp.a  /mnt/etc/rc.d/rc.sysinit
sudo chmod a+x /mnt/etc/rc.d/rc.sysinit
cat > /tmp/tmp.a
# This is the first script to run when startup
::sysinit:/etc/rc.d/rc.sysinit
::restart:/sbin/init
::ctrlaltdel:/sbin/reboot
::shutdown:/bin/umount -a –r
sudo mv /tmp/tmp.a /mnt/etc/inittab
# 创建字符设备
sudo mkdir /mnt/dev
sudo mknod /mnt/dev/tty0 c 4 0
sudo mknod /mnt/dev/tty1 c 4 1
sudo mknod /mnt/dev/tty2 c 4 2
sudo mknod /mnt/dev/tty3 c 4 3
sudo mknod /mnt/dev/console c 5 1
sudo mknod /mnt/dev/null c 1 1
\end{lstlisting}

\subsection{运行qemu}
\begin{lstlisting}[language=bash]
qemu-system-i386  -hda c.img -m 256
\end{lstlisting}

\subsection{卸载磁盘}
在完成上面的操作以后,卸载对应的设备。
\begin{lstlisting}[language=bash]
# 关闭映射设备
$ sudo kpartx -d /dev/loop1
$ sudo losetup -d /dev/loop1
$ sudo losetup -d /dev/loop0
\end{lstlisting}

\section{调试}
\subsection{运行qemu}
\begin{lstlisting}[language=bash]
qemu-system-i386  -hda c.img -m 256 -gdb tcp::1234,nowait,nodelay,server,ipv4 -S
\end{lstlisting}
\begin{itemize}
\item -gdb tcp::1234,nowait,nodelay,server,ipv4 启动gdb调试server
\item -S qemu启动之后就会停止,指导gdb通知其继续运行
\end{itemize}

\end{document}
