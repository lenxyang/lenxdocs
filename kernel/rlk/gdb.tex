\chapter{调试技巧}
\section{使用gdb调试}
\subsection{gdb常用命令}
\subsubsection{查看内存}
\subsection{gdb基本原理}
\subsection{调试信息}

\section{使用qemu跟踪Linux}
qemu提供了kernel选项用于指定内核镜像,通过如下命令linux将直接启动。
\begin{lstlisting}
qemu-system-i386 -kernel bzImage -gdb tcp::1234,nowait,nodelay,server,ipv4 -S
\end{lstlisting}

\section{使用bochs内置调试器}
bochs提供了一个内置调试题,只需要在配置文件当中添加如下一行便可以启动调试器了。带有调试器的bochs一开始运行便会自动定下,而后等待用户命令。
既然有了功能强大且通用的调试器gdb,为什么还需要bochs的内置调试器呢?它确实有着不可替代的作用,尤其是在刚开始入手的时候,内置调试的功能是gdb无法替代的。1
\subsection{查看内存}
\subsection{查看gdt}
\subsection{查看堆栈}
\subsection{日志}
