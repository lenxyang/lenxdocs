\chapter{交叉编译}
仅仅阅读内核是远远不够的,如果能够动态跟踪内核的执行路径,对于理解内核有莫大的帮助。鉴于我们的内核主要面向i386架构,而时下流行的机器及操作系统全部是64位系统,读者是没有办法直接编译32位内核的,必须使用一些功能才能做到。在这方面的投资绝对是值得的,随着对内核理解的加深,如果希望在手机等移动设备平台上调试内核,这些工具都是要熟悉的。

\section{crosstool-ng}
crosstool-ng也是GNU的开源项目,它能够自动下载,编译并生成cross compiler所需要的所有工具,库及操作系统。值得一提的是,某些代码并不能直接编译,需要打些patch,这些工作crosstool-ng都能够自动完成。

\begin{lstlisting}
ct-ng menuconfig
ct-ng build.8
\end{lstlisting}

\section{依赖项}
在开始交叉编译之前,需要安装一系列的编译工具,它们包括



这些工具很容易获取,无论是在linux平台还是在cgywin上,它们都是很基础的包。

\section{常见错误}
编译工具如果在cgywin平台上编译可能会出现大小写的问题,默认情况下windows的文件名不是大小写敏感的,而某些开源软件的代码文件必须要保证这点,否则将会出现文件被覆盖等错误。好在微软并没有堵死“大小写敏感”,用户可以通过注册表来修改“大小写敏感”选项。具体的注册表键值设置如下:
\begin{lstlisting}
 Windows Registry Editor Version 5.00

 [HKEY_LOCAL_MACHINE\SYSTEM\CurrentControlSet\Control\Session Manager\kernel]
``obcaseinsensitive''=dword:00000000
\end{lstlisting}








