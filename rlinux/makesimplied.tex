
\section{简化kbuild}
本着山寨的精神,Linux的Makefile也是需要改造一把的。不是为了让他变得更好,只是为了让它变得更简单,更容易理解。下面的章节将介绍kbuild系统的如何实现,当然那些高级配置部分的实现被省去了,作为一个仅仅支持x86的简陋操作系统,实在不需要配置啊。

\subsection{kbuild基本框架}
kBuild Makefile使用多Makefile来协助管理整个项目。所谓多Makefile,就通过主Makefile引入其他的Makefile文件或者调用其他Makefile的方式来BUILD。kbuild Makefile同时使用了两种方式,它的Makefile定义在根目录,通过引入scripts/Makefile.build,scripts/Makefile.lib,scripts/Makefile.clean作为库。除此之外,每一个代码目录下都包含对应的Makefile,这些Makefile定义了当前路径下包含哪些代码文件需要编译,此外还定义了需要递归调用哪些子目录,这些子目录下也有Makefile,它们与父目录的Makefile完成相同功能。所有这些源代码目录下的Makefile,均通过变量的方式来定义递归调用的子目录,以及需要编译的代码。顶层目录对每一个目录下的Makefile都会调用一下make,而后生成built-in.o(如果编译进内核)或者lib.a(如果作为静态库进行编译)。最后Makefile会将所有的built-in.o及lib.a链接成为vmlinux文件。与BUILD过程类似,clean也采用类似的管理方式。在接下来的一节,我们会尝试简化kbuild系统,仅仅保留它的骨架,让大家对它的工作方式有一个全面的认识。

\subsubsection{掠影}
如果

\subsubsection{目标变量}
kbuild将vmlinux多个目标,它们通常都是以***-y,*-m为名,以-y为结尾的目标会被链接到vmlinux当中,而已-m为结尾的目标为被链接成独立的可载入的文件,很多驱动程序都是以这种方式存在的。此外有很多的模块可供用户选择,它们既可以直接链接进vmlinux也可以当作可载入目标构建。kbuild通过变量在达到目的,例如
\begin{lstlisting}[language=make]
lib-(CONFIG_AAA) += a.o
\end{lstlisting}
变量CONFIG\_AAA可以是空,也可以是y,或者m,在用户配置内核选项的时候会生成配置文件,配置文件当中会指定这些变量的值,在编译过程中make程序便知道将它编译进内核或是作为独立模块了。

下面这段Makefile因子我们的系统,它与linux的Makefile非常相似,不同的是它将大量与主干无关的内容去掉了。
\begin{lstlisting}[language=make]
init-y          := init/
drivers-y       := drivers/ sound/
net-y           := net
libs-y          := lib/

include $(srctree)/arch/$(ARCH)/Makefile

vmlinux-dirs    := $(patsubst %/,%,$(filter %/, $(core-y) \
                $(libs-y)       \
                ))

init-y          := $(patsubst %/, %/built-in.o, $(init-y))
core-y          := $(patsubst %/, %/built-in.o, $(core-y))

vmlinux-init    := $(head-y) $(init-y)
vmlinux-main    := $(core-y) $(libs-y)
vmlinux-all     := $(vmlinux-init) $(vmlinux-main)
vmlinux-lds     := $(srctree)/arch/$(ARCH)/kernel/vmlinux.lds

.PHONY: $(vmlinux-dirs)
$(vmlinux-dirs): scripts
        @echo ``vmlinux-dirs: $(vmlinux-dirs)''
        $(MAKE) $(build)=$@
\end{lstlisting}
接下来我们讲解一下具体的内容,首先Makefile定义了通用框架的内容,而后引入了与架构相关的Makefile,这个Makefile定义了一些变量,比如core-y就是在这个文件定义的,此外它还通过libs-y和core-y通知内核还有那些目录需要递归解析。在所有的路径全部定义好了之后,便可以通过make来递归调用Makefile了,具体规则如下段代码:
\begin{lstlisting}[language=make]
.PHONY: $(vmlinux-dirs)
$(vmlinux-dirs): scripts
        @echo ``vmlinux-dirs: $(vmlinux-dirs)''
        $(MAKE) $(build)=$@
\end{lstlisting}
其实这里的递归调用跟程序的递归调用非常不同,它是通过\$(vmlinux-dirs)这个变量完成的。这个变量定义了所有的需要调用的Makefile路径,由于涉及多个路径它实际上会被拆成多个调用规则。如果将变量展开,那么它看起来会更清楚:
\begin{lstlisting}[language=make]
arch/i386/: scripts
    $(MAKE) $(build)=arch/i386/

init/: scripts
    $(MAKE) $(build)=init

drivers/: scripts
    $(MAKE) $(build)=drivers/i386/

net/: scripts
    $(MAKE) $(build)=net

\end{lstlisting}
通过参数产地所有需要被调用的Makefile的路径全部定义在了变量\$(vmlinux-dirs)当中,make会自动的将所有目录展开,而后逐个调用Makefile。不过\$(MAKE) \$(build)这这句并不是很简单,变量\$(build)实际上是另外一个Makefile,在linux下它对应文件scripts/Makefile.build,在我们的系统中,我将它改成了scripts/build.mk,这仅仅出于个人习惯,请见谅。

除了上面的方式,kbuild还提供了另外一点中递归调用方式,这种方式仅仅是调用make,但并不传递或者改变变量的值,参考下面代码(摘自scripts/build.mk)。
\begin{lstlisting}[language=make]
.PHONY: $(subdir-ym)
$(subdir-ym):
        $(Q)$(MAKE) $(build)=$@

如果在子目录下定义变量subdir-ym,那么make会递归调用对应的subdir-ym。
\end{lstlisting}
\subsubsection{目标定义}
让我们再重新回到主Makefile,大家会注意到
\begin{lstlisting}[language=make]
init-y          := $(patsubst %/, %/built-in.o, $(init-y))
core-y          := $(patsubst %/, %/built-in.o, $(core-y))
\end{lstlisting}
这两句话的意思便是对所有以'/'结尾的目标增加一个后缀,/build-in.o。这几句定义了生成的目标,每个源代码目录生成对应的源文件全部编译成.o文件以后,kbuild会将他们合并为一个文件built-in.o,如果包含库文件还会生成lib.a。

\subsubsection{隐含规则}
读者可能已经看到每一下目标总是.o文件,但linux代码当中却包含多种源代码文件,如.c,.S等,make通过隐含规则(implicit rules)来生成它们。一般来说,默认的隐含规则足够了,但是内核需要指定很多编译选项,因此必须自己指定默认规则。这些规则在scripts/Makefile.build(我们的系统是scripts/build.mk)。

\begin{lstlisting}[language=make]
# 隐式规则列表
%.o: %.c FORCE
        $(CC) $(c_flags) -c -o $@ $<

%.s: %.S FORCE
        $(CPP) $(a_flags)   -o $@ $<

%.o: %.S FORCE
        $(CC) -Da $(a_flags) -c -o $@ $<

%.lds: %.lds.S FORCE
        $(CPP) $(cpp_flags) -D__ASSEMBLY__ -o $@ $<
\end{lstlisting}

\subsubsection{编译及链接选项}
在默认规则当中有很多编译选项,如\${c\_flags},这写规则是根据linux的编译规则产生的,它支持通用(general),针对架构的(arch specified)、针对类别(class specified)及针对文件(file specified)四种编译选项。

\subsubsection{链接文件的顺序}
这里顺便提一句,对于.o文件来说,顺序不是一个太重要的事情,因为GNU ld对于.o文件是重复扫描的,在ld寻找符号的过程中它可能从前面的.o文件当中寻找。

但对于.lib文件来说,顺序就非常重要了,默认情况下ld仅仅顺序扫描,也就是说如果出现一个未定义符号,ld会从后面的.lib当中寻找,如果找不到就会出现undefined symbol错误。举个例子,假设我们有3个上述a, b, c分别定义在库文件a.lib, b.lib和c.lib当中。它们的调用关系为a调用b, b调用c。也可以用依赖关系来表达它们: a依赖b, b依赖c。
如果我们希望将a.lib,b.lib,c.lib链接成为一个可执行程序,那么必须以a b c的顺序出现
\begin{lstlisting}{language=bash}
  ld a.lib b.lib c.lib -o exe
\end{lstlisting}
如果以其他顺序出现,那么ld会报undefined symbol错误。这是因为ld总是认为找不到的符号会在后面的文件出现,它是不会向前查找的。

有时候lib的链接顺序实在难以控制,因此GNU LD提供了编译选项--start-group libs --end-group,在这两个之间出现的lib, GNU ld将采用与.o文件处理类似的方法,重复扫描,与顺序无关。但也会有一个不还好的地方,它会降低ld的性能。

\subsection{系统镜像}
i386架构下制作可以启动磁盘镜像的代码在arch/i386/boot下,此目录下的Makefile比较特别,因为当中的代码并不属于内核,而是属于启动及加载部分。

\subsection{稍许改造}
\subsubsection{sysmap}
为了调试方便Linux在生成玩vmlinux之后,还通过nm命令生成了一张记录所有符号地址的表保存在文件当中。我们不打算这么做,毕竟去查表也是挺复杂的一件事情,为了方便起见我们还是使用gdb调试。因为我们在链接阶段生成两个版本的vmlinux。一个包含调试信息,它仅仅用来帮助调试器识别符号及变量。另外一个文件不太调试信息它将被写入到磁盘当中作为操作系统运行。无论生成那种文件,在编译阶段都通过选项-g使其包含调试信息。在链接不包含调试信息的版本是用 ld -s(或者--strip-all) 生成即可。--strip-all选项会取出所有符号相关的信息。
