\chapter{GNU Toolchain}
\section{GNU AS简介}
\section{GNU AS}
\section{机器码编程}
GAS是一种跨平台的语言,在汇编阶段,它会将程序转换为对应机器架构的代码。为了支持这种跨平台,它的语言规范必须支持多个平台。但某些平台可能没有一些指令操作,GAS就不得不舍弃了。在这里我们介绍一个最基本的指令ljmp,它非常常用,但GAS的ljmp支持的指令格式非常少,这对我们早场了一定的麻烦,举个最简单的例子,在x86保护模式下,如果我们希望跨段调用,可以运行
\begin{lstlisting}
  jmp seg:off
\end{lstlisting}
对应的GAS指令是。
\begin{lstlisting}
  jmp seg,off
\end{lstlisting}
但它还有一个格外的限制,就是seg和off必须是常熟,不能是寄存器也不能是内存,那么如果我们预先不知道入口地址呢?(比如加载elf文件,它的入口地址保存在elf文件当中,我们必须从文件里读出并保存在寄存器或是内存当中,而后跳转到这个值,程序才能运行)。
此外,对于intel来说,如果包含16bit和32bit的代码,也要使用这种方式,GAS对于某些intel的指令前缀可能支持的不够好,如下:arch/i386下采用的方式是直接使用机器码,参见arch/i386/boot/setup.S,这段代码将进入到vmlinux并执行。
\begin{lstlisting}
# jump to startup_32 in arch/i386/boot/compressed/head.S
#
# NOTE: For high loaded big kernels we need a
#       jmpi    0x100000,__BOOT_CS
#       but we yet haven't reloaded the CS register,
#       so the default size of the target offset
#       still is 16 bit.  However, using an operand 
#       prefix (0x66), the CPU will properly take our 
#       48 bit far pointer. (INTeL 80386 Programmer's 
#       Reference Manual, Mixing 16-bit and 32-bit code, 
#       page 16-6)

        .byte 0x66, 0xea   # prefix + jmpi-opcode
code32: .long   0x1000     # will be set to 0x100000
                           # for big kernels
        .word   __BOOT_CS

\end{lstlisting}
上面这段代码之所以写成这样,就是由于在16bit模式运行的CPU, 寄存器CS也被看作16bit的,此时我们没有没法访问高于高于64K的内存,因此必须使用intel的16-bit,32-bit混合编程技巧了,在Intel Software Developer's Manual的第三卷当中有单独的一张来介绍这个技巧:通过给指令增加前缀0x66达到目的。

同样适用类似的技巧,我们还可以在调用指令之前修改目标地址,从而使得我们可以指定动态确定跳转的目标。
\subsection{Intel指令格式}
\section{GNU C嵌入汇编}
\section{汇编语言调试技巧}
