\chapter{系统加载}

在上一章我们的程序已经取得了机器的控制权,但是他只能做些无聊的事情:打印些东西而后空转。本章它将开始并完成的它的历史使命“加载内核并运行“,不过这个内核会它一样无聊,还是打印些东西然后空转。不过这确实巨大的进步,因为我们的程序已经从汇编变成了带有C的汇编,而且被加载的程序不再受到512字节的限制,它有着无限的可能性。
\section{PC启动程序}
\subsection{读磁盘}
\subsection{ELF文件格式}
\subsection{准备一个ELF格式的内核}
\subsection{开启20位寻址}
系统开机以后,CPU运行于实模式下且仅仅支持16位寻址。这就意味着我们最多可以使用64K的内存。对于bootsect.S的内部处理这个空间应该够用了,但如果想要加载vmlinux这个巨大的程序可以远远不够的。为此我们需要将寻址空间提高的20位,即1M。这也意味着我们的操作系统最多也就只能有1M的大小,这对于一个没有什么驱动程序的操作系统来说足够用了。
\subsection{加载}
为了能够将内核加载到高于64K的内存,我们付出了很多努力,与对任何一个超过64K的内存,我们必将将它分成两个部分存储。实际上,实模式也是可以访问4G物理内存的。接下来我们做个实验看看如何做到这一点,不然的话,我们怎么能像Linux那样将内核加载到1M内存之上呢,而且vmlinux的可是很大的一个文件,1M的寻址空间一定是不够的。
\subsection{打开4G内存访问}
实际上,每一个段寄存器都有可以分两个部分,可见部分(visible part)和隐藏部分(hidden part)。其中隐藏部分有时候叫做descriptor cache或者shadow register。后面保护模式章节会提到,保护模式下段寄存器保存的是selector,读者可以简单的理解成数组的下标,在每次使用selector的时候,CPU都需要查表。这明显会影响效率,CPU提供了一种机制在查表的时候顺便将GDT的值存放在段寄存器的不可见部分,因此它又叫做descriptor cache,因此段寄存器部分的格式就是descriptor的格式,它的具体格式如下图所示:
\begin{pspicture}
\end{pspicture}
这里仅仅介绍与内存相关的部分就好,其它的细节还会在“保护模式“当中详细介绍。

通过修改在修改段寄存器具体内容的时候,它的隐含部分也会更改。利用这种机制可以实现4G的

修改CS隐含部分的值需要使用模式切换,但CPU进入到保护模式的时候,段寄存器的隐含部分自动改变它的状态,但从保护模式且会real-mode,段寄存器的属性则不会改变,我们就利用这个技巧使boot程序能够访问4G内存。
还是通过代码来说明如何打开real-mode下的4G访问
\begin{lstlisting}
        .code16
        .text

.equ    __CODE32_CS,    0x8
.equ    __DATA32_DS,    0x10
.equ    __CODE16_CS,    0x0

        .global _start
_start:
        mov     %cs, %ax
        mov     %ax, %ds
        mov     %ax, %ss
        mov     $0x7c00, %sp

        lgdt    gdtptr

        /* 打开保护模式 */
        mov     %cr0, %eax
        or      $0x1, %eax
        mov     %eax, %cr0

        /* 进入保护模式 */
        .byte   0x66, 0xea
        .long   pm
        .word   0x8

        .code32
pm:
        mov     $__DATA32_DS, %eax
        mov     %eax, %ds
        mov     %eax, %fs
        mov     %eax, %gs

        /* 关闭保护模式 */
        mov     %cr0, %eax
        and     $0xFFFFFFFE, %eax
        mov     %eax, %cr0

        .byte   0x66, 0xea
        .long   real
        .word   __CODE16_CS

        .code16
real:
        mov     $0x7c00, %sp

        mov     $1234, %eax
        mov     $0x100000, %edx
        mov     %eax, (%edx)

        .loop:
        jmp     .loop
gdt:
        .quad   0x0000000000000000 /* NULL selector */
        .quad   0x00cf9a000000ffff /* base=0, limit=0xffff, DPL=0 */
        .quad   0x00cf92000000ffff /* base=0, limit=0xffff, DPL=0 */
gdtptr:
        .word   gdtptr - gdt
        .long   gdt
        .word   0
.org    510
        .word   0xAA55
\end{lstlisting}

\subsubsection{验证}
如何验证呢? 我们不用程序的方式来验证,该有调试器的方式验证,这样做可以避免由于我们理解不到位造成错误。试想如果对内存的理解方式有问题,按照同样的方式写入熟读到内存然后再按照同样的方式读出内存,那结果是一样的,但我们真的不数据写入到了我们希望的那段内存了吗?如果发生了“×××”怎么办,我们根本不知道。比较好的方法就是使用两种不同方式,比如使用自己的程序写入内存,而后用内置调试器来访问物理内存验证数据,这种方法可以确保万无一失。
接下来将使用bochs内置调试器来验证,这样做的原因很简单,gdb远程调试的方式无法获得CPU的状态,某些状态gdb根本就不支持,比如gdt的内容、sreg的隐含部分都无法通过gdb来查看。bochs专门针对x86平台开发,有强大查看CPU内核状态的工具。
\begin{lstlisting}[language=bash]
<bochs:> break *0x7c00
<bochs:> c
<bochs:> n...
<bochs:27> sreg
es:0x0000, dh=0x00009300, dl=0x0000ffff, valid=1
        Data segment, base=0x00000000, limit=0x0000ffff, Read/Write, Accessed
cs:0x0000, dh=0x00cf9300, dl=0x0000ffff, valid=1
        Data segment, base=0x00000000, limit=0xffffffff, Read/Write, Accessed
ss:0x0000, dh=0x00009300, dl=0x0000ffff, valid=7
        Data segment, base=0x00000000, limit=0x0000ffff, Read/Write, Accessed
ds:0x0010, dh=0x00cf9300, dl=0x0000ffff, valid=7
        Data segment, base=0x00000000, limit=0xffffffff, Read/Write, Accessed
fs:0x0010, dh=0x00cf9300, dl=0x0000ffff, valid=1
        Data segment, base=0x00000000, limit=0xffffffff, Read/Write, Accessed
gs:0x0010, dh=0x00cf9300, dl=0x0000ffff, valid=1
        Data segment, base=0x00000000, limit=0xffffffff, Read/Write, Accessed
ldtr:0x0000, dh=0x00008200, dl=0x0000ffff, valid=1
tr:0x0000, dh=0x00008b00, dl=0x0000ffff, valid=1
gdtr:base=0x0000000000007c51, limit=0x18
idtr:base=0x0000000000000000, limit=0x3ff
\end{lstlisting}

再向后过 mov \%eax, (\%edx)之后,验证内存是否已经写入
\begin{lstlisting}[language=bash]
bochs:31> x 0x100000
[bochs]:
0x0000000000100000 <bogus+       0>:    0x000004d2
\end{lstlisting}
4d2为1234的16进制表示,我们确实成功了。

\subsection{真实世界的vmlinux}
为了给大家呈现一个简单的世界,作者自己实现了一个内核加载程序。这个程序加载简化的内核足够用了,但对于真实的linux内核,他是绝对不够用的。原因很简单,vmlinux如果包含调试信息的话有200M那么大,在实模式下是绝对无法加载的。真实世界的linux通常是由GRUB加载,当然还有其他的加载器。GRUB基本上已经是一个小型的操作系统了。它支持与Linux相同的驱动程序框架,并且包含了很多驱动程序,比如ext2,msdos等文件系统的驱动。当然为了统一支持他们,还需要VFS。此外还携带一个shell。当然它不支持操作系统的高级特性,比如复杂的内存管理,多进程的切换和调度,它都没有。但它一定是运行在保护模式下的,在完成加载之后回到实模式,而后将控制权交给目标操作系统。在我们的简化操作系统基本成型以后,我们会尝试使用GRUB2来加载我们的系统。
\section{调试技巧}

程序已经可以运行了,接下来我们熟悉一些调试技巧,这对我们将来开发非常有用。俗话说程序是三分编七分调。程雪写完也就是完成了三成的工作而已,真正复杂的是调试呕。

只要是程序就一定有bug。尤其是按照前面我们所鼓励的方式:将代码抄进文件编译运行。如果有时间作者强烈推荐这种方式,它能够增强你对系统的理解。但这种方式一定会让你的程序存在bug,之前已经介绍了qemu自身就带有gdbstub。让我们一起感受他的强大吧。
\subsection{gdb远程调试}
\begin{lstlisting}[language=bash]
 symbol-file out/arch/i386/boot/bootsect.gdb
 set architecture i8080
 target remote :1234
 hbreak *0x7c00
\end{lstlisting}

\subsection{symbol-file}
前面介绍过,启动程序必须控制在512字节,而且它必须是binary格式的。所谓binary格式就是CPU可以从第一个字节开始运行的程序。因此调试信息是不可能存在与binary之上的。不过大家不用担心,因为我们还生成了一个ELF文件,这个文件包含完整的调试信息。
调试信息大概是什么样子呢?objdump可以帮助大家来理解调试信息看起来是什么样子?正如刚才提到,make之后生成了两个文件bootsect.gdb和bootsect.bin,bootsect.gdb包含完整信息,而bootsect.bin仅仅包含程序。我们用objdump来查看反汇编一下bootsect.gdb的内容,大家可以看到
\begin{lstlisting}[language=bash]
$ objdump.exe  -dl arch/i386/boot/bootsect.gdb

arch/i386/boot/bootsect.gdb:     file format elf32-i386


Disassembly of section .text:

00007c00 <_start>:
arch/i386/boot/bootsect.S:17
    7c00:       8c c8                   mov    %cs,%eax
arch/i386/boot/bootsect.S:18
    7c02:       8e d8                   mov    %eax,%ds
arch/i386/boot/bootsect.S:19
    7c04:       8e c0                   mov    %eax,%es
arch/i386/boot/bootsect.S:20
    7c06:       8e e0                   mov    %eax,%fs
arch/i386/boot/bootsect.S:21
    7c08:       8e e8                   mov    %eax,%gs
arch/i386/boot/bootsect.S:23
    7c0a:       66 bc f0 7f             mov    $0x7ff0,%sp
    7c0e:       00 00                   add    %al,(%eax)
arch/i386/boot/bootsect.S:24
    7c10:       66 31 ed                xor    %bp,%bp
arch/i386/boot/bootsect.S:29
    7c13:       fa                      cli
arch/i386/boot/bootsect.S:30
\end{lstlisting}
这就是调试信息,它将源代码文件与地址对应起来。在我们进行反汇编的时候,它用显示源代码而非反汇编代码。那么反汇编代码看起来如何呢?

\begin{lstlisting}[language=bash]
$ objdump.exe  -d arch/i386/boot/bootsect.gdb

arch/i386/boot/bootsect.gdb:     file format elf32-i386


Disassembly of section .text:

00007c00 <_start>:
    7c00:       8c c8                   mov    %cs,%eax
    7c02:       8e d8                   mov    %eax,%ds
    7c04:       8e c0                   mov    %eax,%es
    7c06:       8e e0                   mov    %eax,%fs
    7c08:       8e e8                   mov    %eax,%gs
    7c0a:       66 bc f0 7f             mov    $0x7ff0,%sp
    7c0e:       00 00                   add    %al,(%eax)
    7c10:       66 31 ed                xor    %bp,%bp
    7c13:       fa                      cli
    7c14:       e4 92                   in     $0x92,%al
    7c16:       0c 02                   or     $0x2,%al
    7c18:       e6 92                   out    %al,$0x92
    7c1a:       fb                      sti
    7c1b:       e8 82 01 be ba          call   babe7da2 <ELFDATA_ADDR+0xbabc7da2>
    7c20:       7d e8                   jge    7c0a <_start+0xa>
    7c22:       6c                      insb   (%dx),%es:(%edi)
    7c23:       01 b8 00 20 8e c0       add    %edi,-0x3f71e000(%eax)
    7c29:       bf 00 00 b8 01          mov    $0x1b80000,%edi
    7c2e:       00 b9 32 00 e8 29       add    %bh,0x29e80032(%ecx)
    7c34:       01 e8                   add    %ebp,%eax
    7c36:       08 00                   or     %al,(%eax)
    7c38:       66 ea 00 00 10 00       ljmpw  $0x10,$0x0
    ...
\end{lstlisting}
我们大致知道调试信息就是把源代码文件和可执行文件对应起来,那么它有什么用呢?还是举例说明吧,比如说我们希望在跳入到内核之前就是指令ljmp处设置断点。如果没有调试信息,我们必须首先返回便代码找到对应的地址 0x7c38,然后用gdb在物理内存0x7c38处设置断点;如果gdb加载了调试信息,同样是对ljmp设置断点,我们只需知道所处的文件行处就可以啦,hbreak bootsect.S:55,是不是省去了很多麻烦。此外,在我们执行执行step之类指令的时候,有调试信息是我们能够使用GUI工具跟踪当前执行到了那一行源代码,源代码可是有注释的;如果没有调试信息,你就能看到枯燥的汇编代码了。
当然实际上调试信息包含的内容更多,如果想了解更多内容可以参考gdbstub相关的文档。

\subsection{bochs内置调试}
gdb调试器非常强大,但它有一个致命缺点:它非常的主观。它到底有多主观呢,这完全取决于用户。在gdb调试的时候,虚拟机会提供一部分数据,比如内存,如果用户想要查看内存当中到底是什么指令,就可以调用diassemble命令来反编译。但gdb并不知道现在CPU处于什么状态,因此它会根据用户的设置来反汇编。
\begin{lstlisting}[language=bash]
# 告诉gdb按照16bit的指令来反汇编
set architecture i8086
# 通知gdb按照32bit模式来返回编
set architecture i386
\end{lstlisting}

你经常会发现明明执行了一条指令,但结果却与预期不同,这就是反汇编错误导致的。

bochs的内置调试器却绝对不会犯类似的错误,它会跟踪CPU的状态,当进入保护模式以后他就会按照i386模式反汇编,如果跳出保护模式则按照i8086的模式反汇编。这种可观性能够帮助我们很快的找到错误。用上面realtime4G这个程序作为例子,我们将gdt的内容做一下更改:
\begin{lstlisting}[language=bash]
gdt:
        .quad   0x0000000000000000
        .quad   0x00cf9a000000ffff
        .quad   0x00cf92000000ffff
\end{lstlisting}
更改以后整个段编程了16禁止的,这时候数据段编程了16bit模式的,运行至返回式模式处的代码而后继续运行,查看它们的代码

\begin{lstlisting}[language=bash]
<bochs:21>
Next at t=14041024
(0) [0x0000000000007c42] 0000:7c42 (unk. ctxt): add byte ptr ds:[eax], al ; 0000
<bochs:22>
Next at t=14041025
(0) [0x0000000000007c44] 0000:7c44 (unk. ctxt): mov ax, 0x04d2            ; 66b8d204
<bochs:23>
Next at t=14041026
(0) [0x0000000000007c48] 0000:7c48 (unk. ctxt): add byte ptr ds:[eax], al ; 0000
<bochs:24>
Next at t=14041027
(0) [0x0000000000007c4a] 0000:7c4a (unk. ctxt): mov dx, 0x0000            ; 66ba0000
<bochs:25>
Next at t=14041028
(0) [0x0000000000007c4e] 0000:7c4e (unk. ctxt): adc byte ptr ds:[eax], al ; 1000
<bochs:26>
Next at t=14041029
(0) [0x0000000000007c50] 0000:7c50 (unk. ctxt): mov word ptr ss:[bp+si], ax ; 67668902
<bochs:27>
Next at t=14041030
(0) [0x0000000000007c54] 0000:7c54 (unk. ctxt): jmp .-2 (0x00007c54)      ; ebfe
\end{lstlisting}
看到了吗,与我们的预期完全不同,很快我们就能找到错误,原味数据段被指定成了32bit的了。实际上,这种问题在gdt当中是比较常见的,在刚开始学习保护模式编程的时候,bochs的内部调试器真的很有帮助,建议读者这两种调试器都要熟悉才好。
