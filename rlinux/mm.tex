
\chapter{Linux内存管理}

\section{进入保护模式}
在进入开始介绍内存管理之前,我们需要让我们的系统进入保护模式并开启分页机制,虽然这部分内容如果使用C语言来写应该非常的简单,但这部分内容确实使用汇编语言实现,这绝对不是为了让读者看起来很cool。采用这种技术是有原因,理解这个原因是非常重要的。

在保护模式下,程序有了4G的寻址空间(其实这个问题比较复杂,很难一两句说清楚,在内存管理部分我们再详细的讲)。Linux内核将4G空间划分成了两个部分,用户控件(0G~3G)及内核空间(3G~4G)。vmlinux在加载的时候其实并不位于3G~4G空间之上。
原因包括多个:首先PC机并未启动分页机制,它能访问的是物理内存,而无力内存有没有4G很难说。第二及时真的有4G内存,在进入保护模式之前,16位的程序也没有办法访问到这些物理内存。

因此Linux采用了一个很精巧的办法,将vmlinux加载到低端内存。再启动分页机制后将它映射到高端内存上。这对C程序来说绝对是个挑战,因为从链接的时刻各个符号的地址已经确定了,如果要写两个模式下运行的程序,每看到一个符号叫要想想现在处于什么模式,究竟是以物理内存还是用虚拟内存的方式访问这个符号。如果是以物理内存的方式访问,还要进行指针运行而后在进行类型转换,想想都头大。

简单的做法是写一段汇编代码,这段代码之后就进入分页模式,C程序向怎么写就怎么写;再次之前就要讲过一些运算了,不过这段代码比较短,整体还是可以接受的。由于进入内核之前必须先执行这段代码,因此这段代码的名字就叫做head.S(完整路径是arch/i386/kernel/head.S)。我们的代码做了很大简化,感兴趣的同学可以对比,具体代码如下:
\begin{lstlisting}

	
\end{lstlisting}
这部代码就是Linux进入保护模式的代码,从当中可以看到Linux包含两部分的

当中还是包含很多将来要用的东西的,比如cpu\_gdt\_table。 在系统初始化阶段Linux采用boot\_gdt\_table。在正式启动之后将会切换到cpu\_gdt\_table上来。

好了到现在为止我们的系统已经进入了保护模式,接下来我们逐个初始化操作系统的子系统。


\section{构建新的内核}

内核是有汇编和C联合编写的,与寻常的程序不同,我们不能简单的ld $(LDFLAGS) fileslist -o vmlinux,而使使用一种叫做ldscript的脚本。 ld script提供了非常有用的功能,使我们在编写内核的时候更简单:
1. 制定不同段的位置
2. 将多个段合并成一个段
3. 初始化全局变量

在启动分页机制的时候大家注意到了变量pg,它保存了全局页表,当将它放在那里呢?最好的位置是程序的尾部,但在链接之前我们是不知道程序的尾部在那里,不过没关系,ld script可以告诉我们。
\begin{lstlisting}[language=bash]


\end{lstlisting}


\subsection{ldscript}
ldscript非常的容易

\subsection{模仿Linux的机制}
Linux调用lds并不像我们介绍的那么简单直接,因为它真的做了一些使的代码更容易维护的事情。但为了提高维护性,我们必须提高我们对工具的使用能力才行。
首先来介绍一下linux使用lds的方式,它并不直接使用lds文件,而是使用一种叫做lds.S的文件之后来生成lds文件,为什么这么作呢,因为linux希望所有的变量都统一的放在一个地方,而不是分散在各地。但是lds是不支持包含C语言的头文件的。但是.S文件支持,即GAS格式的汇编语言文件。

GAS格式的汇编文件有两种.s和.S,这两种文件有一个微小的不同,虽然微小但足够让你崩溃一天。.s和.S的文件都支持gcc或者as进行汇编,但是.s的文件是直接忽略include指令的,是的直接忽略还不报错,让你误以为引入了都文件但实际上没有。如果你在你的汇编程序中用到了某个\#define定义的常量,它也不报语法错误,而是报undefined symbol。这个错误如此具有误导性,以至于我会怀疑C语言通过\#define定义的常量会包含在symbol table当中。但实际上完全不会,它仅仅是没有包含头文件而已,把扩展名换成.S立时所有的问题都没有了。


\section{Linux Memory Management}
接下来主要介绍Linux内存管理。

Linux支持多种体系结构,为了方便这种跨体系结构的开发, Linux使用了大量的类似于C++ Virtual Function的方式。
首先它为每一个子系统定义了一组通用操作,在不同的系统下有不同的实现。
Linux在编译期间可以根据宏定义来include对应的头文件编译对应的c文件。
MM(memory management)与Linux其他模块一样也是按照这种方式组织的, 首先在根目录mm下定义了通用的操作及通用的逻辑实现。
与平台相关代码则放在了arch/i386/mm下, 这其中包括i386 mm的初始化及差异化的功能实现。

我们从i386的内存分页机制及初始化流程来开始,一步一步分析,一步一步实现。
在基础打好之后,我们在研究更高层的linux内存管理策略。


\section{i386体系下的内存管理}


处理器所支持的RAM容量收到地址总线的限制,早期Intel(80386)的处理器虽然有32位地址总线,但是由于某些原因,它仅仅能够提供1G的寻址能力。
这对于现在的应用程序来说显然是不够用的。
Intel通过增加地址引脚从32位增加到了36位,使寻址能力达到了64GB, 之后Intel引入了PAE(Physical Address Extension)的机制,使得CPU的选址能力达到了这个范围。

\subsection{i386分页基础}
i386分页机制提供了一种从线性地址向物理地址的映射功能。
在程序试图访问线性地址时,CPU的MMU自动计算出对应的物理地址。 
这种机制是应用程序寻址空间隔离的基础。

MMU是如何计算出与线性地址对应的物理地址的呢? 这个工作由CPU内部的MMU来完成。
简单的来说MMU完成的工作就是将一个完整的32位地址分段查表最终得到对应的物理地址。
以三级页表为例,首先我们一直全局页表(PGD)的地址保存在一个特殊寄存器CR3当中。
当给定一个32位地址的时候,我们根据31~21(共10位地址)来计算第二级页表的起始地址,直观一点用代码来表示就是\begin{lstlisting}[language=C]
  pmd = pgd + (vaddr & 0xFFC0000) >> 22)
  pte = pmd + (vaddr & 0x003FF000 >> 12)
  paddr = *pte + vaddr & 0x00000FFF
\end{lstlisting}

物理内存地址等于对应表项存储的值, 当中由于表项是4K对齐的PTE 当中保存的其实是页的地址, 最终还要加上页内的便宜地址得到对应的物理地址。

\subsection{多级分页支持}

Intel提供了多级分页的机制,上面介绍的是3级分页, 对于x86-64系统,Intel提供了4级分页。在32位系统下,如果不启用PAE,仅仅对1G空间进行寻址,CPU可以使用2级分页。此外Intel还提供了扩展分页,这是一种以4M作为一页的分页方式。Linux定义了几个概念和结构题用来区别多级页表机制当中各个级别的名字。
\begin{itemize}
  \item PTE(Page Table Entry)
  \item PMD(Page Middle Directory)
  \item PUD(Page Upper Directory)
  \item PGD(Page Global Directory)
\end{itemize}

\subsection{初始化}

\begin{lstlisting}
\end{lstlisting}

其中pg0代表一个安全的位置,即pg0处的内存既没有代码也没有数据,是一块可以用的物理内存。
如何才知道自己程序和数据的尾部处于内存的什么位置呢, GLD(我们常用的链接程序, GNU ld)提供了这么一种能力,我们可以在连接脚本加入一些变量,这些变量会被链接程序当作符号加载到ELF文件当中。
这也就是为啥我们可以在arch/i386/mm/pgtable.h当中看到pg0的声明,却总找不到它定义在那个代码文件当中的原因。

还有一个问题为什么每个值都要减去\_\_PAGE\_OFFSET呢,这个值有代表一个非常大的值。
这个原因也可以从链接脚本当中找到, 在链接脚本的开始出我们可以发现这个脚本指定的其实地址是\_\_PAGE\_OFFSET + 0x100000。
但实际上在初始化阶段我们将它仅仅加载到了0x100000。
之后我们通过映射的方式将它们映射到\_\_PAGE\_OFFSET + 0x100000这里,之后的代码就可以顺利运行了。
通过这机制也可以保证此后所有的内核代码都是处于0xC0000000之上的,与我们之前的约定一致。
在启动虚拟内存之前,我们访问任何一个vmlinux上的符号,都必须使用它的物理地址,即虚拟地址-0xC0000000。
在启用虚拟内存之后,我们就可以完全使用C程序来实现了。

这也就是为什么这么一段简单的程序要使用汇编语言实现,而不用C程序实现了。

常量 \_\_PAGE\_OFFSET定义了内核空间所处的位置,这个位置是虚拟地址,对于32位系统来说是4G当中最高的那1G,即0xC00000000~0xFFFFFFFF。
在Linux下,这部分地址全部交由内核来使用,应用程序是无法直接访问的,而且对于所有的应用程序这部分的地址的内容是一样的。

变量swapper\_pg\_dir是一个有1024项的数组,他就是我们初始化阶段的PGD。

后边的代码就比较容易看懂了,就是不断的填满这1024项。


有一个问题不知道读者发现了没有,这段代码如果使用C语言来实现的话应该是非常容易的是什么原因导致它没有这么做呢?

这段代码执行完毕后,我们已经完成从0开始到8M部分的初始化。

\subsection{FAQ}
为何32位地址总线却只能提供30位的寻址能力?

\section{Init Pagging}

在架构相关的初始化真正开始时,page table的映射才开始建立,他的入口在paging\_init函数。
这个函数定义在文件arch/i386/mm/init.c
\begin{lstlisting}[language=C]
static void __init pagetable_init(void) {
  u32 addr;
  pgd_t *pgd_base = swapper_pg_dir;

  // if PSE available
  kernel_physical_mapping_init(pgd_base);
}

void __init paging_init(void) {
  pagetable_init();

  load_cr3(swapper_pg_dir);

  __flush_tlb_all();

  kmap_init();
  zone_sizes_init();
}
\end{lstlisting}


真正负责页表初始化的是kernel\_physical\_mapping\_init, 他负责PGD, PMD和PTE三级页表的初始化。
它的逻辑非常简单。
需要注意的一定是PGD一定是连续的1024项,但是PMD和PTE就不一定了,如果上一级的页表项没有用,那么下一级根本就不需要分配空间。
这部分细节藏在函数one\_md\_table\_init和函数one\_page\_table\_init当中。
\begin{lstlisting}[language=C]
/*
 * This maps the physical memory to kernel virtual 
 * address space, a total of max_low_pfn pages, by 
 * creating page tables starting from address
 * PAGE_OFFSET.
 */
static void __init kernel_physical_mapping_init(pgd_t *pgd_base) {
  u32 pfn;
  pgd_t *pgd;
  pmd_t *pmd;
  pte_t *pte;

  int pgd_idx, pmd_idx, pte_ofs;

  pgd_idx = pgd_index(PAGE_OFFSET);
  pgd = pgd_base + pgd_idx;
  pfn = 0;

  for (; pgd_idx < PTRS_PER_PGD; pgd++, pgd_idx++) {
    pmd = one_md_table_init(pgd);

    if (pfn >= max_low_pfn) {
      continue;
    }

    for (pmd_idx = 0; pmd_idx < PTRS_PER_PMD && pfn < max_low_pfn;
         pmd++, pmd_idx++) {
      pte = one_page_table_init(pmd);
      for (pte_ofs = 0; pte_ofs < PTRS_PER_PTE && pfn < max_low_pfn;
           pte++, pfn++, pte_ofs++) {
        if (is_kernel_text(address)) {
          set_pte(pte, pfn_pte(pfn, PAGE_KERNEL_EXEC));
        } else {
          set_pte(pte, pfn_pte(pfn, PAGE_KERNEL);
        }
      }
    }
  }
}
\end{lstlisting}


\section{Page分配及释放}
 这一节的名字是内存的分配与释放,其实这已经包含了所有内存管理的内容。
之所以起这个名字而非内存管理,是希望给大家一个动态的感觉:我们是以分配及释放的过程来讲述这部分内容的。

内核本身占用的物理内存并不多,内核将空闲的page都分区域保存起来,这个区域就是zone。
当内核试图分配内存时会挑选合适的区域分配若干个page并对page分配线性地址,之后linux还要更新页表。
这个过程就是Linux分配内存的过程。


整个过程最复杂的部分在于挑选合适的页,这个过程也是分层次的, 最底层以内存大小为目标进行分配,这个系统的叫做“伙伴系统”。
再上一层是“面向对象”的,这个对象从C语言的角度看就是结构体,简单的理解就是建立一组内存缓冲区,缓冲区的key是结构体, 当某个结构体释放时内存将放回到这个结构体的缓冲当中,分配时则尽量从缓冲区分配,否则使用下一层的“伙伴系统”进行分配。
除了slab之外,Linux还有其他策略。
这些策略针对不同体系的机器,各有特斯,以后我们会逐一比较。
从上面的描述的过程来看,这里面涉及到几个核心的概念,我们逐一的介绍。

\subsection{page}
page经常被翻译做页框,它是以4K为单元的物理内存单元(当然也可能是2M,与CPU的分页模式有关)。
每一个page都有一个描述符与之对应,它保存着page的信息。


在内核完成映射之后,我们实际上占用了所有的物理内存。
\subsection{Zone}

\subsubsection{HighMemory}
内核内存从0xC000000 + 0x100000开始,0~3G的内存交给用户控件使用,而32位机器既能够寻址的最大范围是4G,再高的内存就没有办法映射了。
其他的内存就只能放在High Memroy区域了。
这部分内存的特点是,他们没有进行映射,需要的时候才进行映射。

\subsection{page的分配与释放}

