\section{kbuild核心流程}
当用户在linux源代码目录下执行make的时候,构建系统就开始运作了。最开始执行的就是源代码根目录下的Makefile,linux的文档管它叫'top level Makefile'。
简单的说它包括如下共呢
\begin{compactenum}
\item 配置内核并产生.config文件
\item 生成include/linux/version.h文件
\item 根据目标arch的类型建立include/asm软链接
\item 更新所有的依赖
\item 递归的便利所有在init-*, core* drivers-* net-* libs*并build所有目标
\item 将生成的文件链接成vmlinux并存放在源代码根目录下
\item 最后准备用于安装的文件
\end{compactenum}

\subsection{变量初始化}
\subsubsection{LDFLAGS}
它是通用链接选项,所有的ld命令都会是用它
\subsubsection{LDFLAGS\_MODULE}
此选项用于生成可加载模块.ko的ld命令
\subsubsection{LDFLAGS\_vmlinux}
顾名思义,它是用于链接vmlinux的选项
\subsubsection{OBJCOPYFLAGS}
某些链接器不支持binary格式,此时可以通过objcopy命令来实现。例如
\begin{lstlisting}[language=bash]
OBJCOPYFLAGS := -O binary
# arch/i386/boot/Makefile
$(obj)/image: vmlinux FORCE
    $(call if_changed,objcopy)
\end{lstlisting}
\subsubsection{AFLAGS}
汇编程序的通用选项
\subsubsection{CFLAGS}
通用的C语言编译选项
\subsubsection{CFLAGS\_KERNEL}
生成内核的通用编译选项
\subsubsection{CFLAGS\_MODULE}
生成可加载模块的编译选项
\subsection{递归编译所有subdir}
真正的构建开始了,top level Makefile会按照一定次序递归访问subdir下的Makefile。这些子目录通过以下几个变量定义:
head-y、init-y、core-y、libs-y、drivers-y、net-y
\begin{compactenum}
\item head-y: 所有需要首先放到vmlinux的目标
\item libs-y: 所有用来生成lib.a的目录
\end{compactenum}
这些变量的定义分成两个部分,一部分在top level Makefile, 它定义了所有与平台架构无关的目录。另外一部分与平台相关的则定义在arch/\$\{ARCH\}/Makefile当中。
\begin{lstlisting}[language=make]
core-y += arch/i386/kernel/
libs-y += arch/i386/lib
drivers-$(CONFIG_OPROFILE)  += arch/sparc64/oprofile/
\end{lstlisting}

\subsubsection{链接文件的顺序}
这里顺便提一句,对于.o文件来说,顺序不是一个太重要的事情。但对于.lib文件来说,顺序就非常重要了。举个例子,假设我们有3个上述a, b, c分别定义在库文件a.lib, b.lib和c.lib当中。它们的调用关系为a调用b, b调用c。也可以用依赖关系来表达它们: a依赖b, b依赖c。
如果我们希望将a.lib,b.lib,c.lib链接成为一个可执行程序,那么必须以a b c的顺序出现
\begin{lstlisting}{language=bash}
  ld a.lib b.lib c.lib -o exe
\end{lstlisting}
如果以其他顺序出现,那么ld会报undefined symbol错误。这是因为ld总是认为找不到的符号会在后面的文件出现,它是不会向前查找的。
\subsection{生成vmlinux}

除了构建流程之外,kbuild还有clean流程,这部分比较简单,大家可以参考文档。


\section{Linux目录结构}
\dirtree{%
.1 rlinux.
.2 arch.
.3 i386.
.3 mm.
.3 kernel.
.3 mach-default.
.3 boot.
.2 drivers.
.2 fs.
.2 include.
.3 asm.
.3 asm-i386.
.3 linux.
.2 init.
.2 kernel.
.3 timers.
.2 lib.
.2 mm.
.2 scripts.
}
