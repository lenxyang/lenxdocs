\chapter{x86保护模式}
\section{保护模式基本概念}
为了讲解后面Linux的代码,我们需要首先熟悉一下保护模式的基本概念。
为了简化汇编代码对这些内容的影响,我们尽量使用C代码来展示。
就在刚才经过我们艰苦卓绝的努力,bootsect已经可以支持ELF文件了。
现阶段它最大的好处解释,我们可以摆脱单纯的汇编而使用C语言进行编程了。
这绝对是一个生产力的巨大飞跃。

\section{最简单的保护模式程序}

\section{进入保护模式}
加下来我们就要在自己的内核当中添加进入保护模式的代码了。很不幸的是它是使用汇编写的。
这当然是有原因的,理解这个原因是非常重要的。
在保护模式下,程序有了4G的寻址空间(其实这个问题比较复杂,很难一两句说清楚,在内存管理部分我们再详细的讲)。
Linux内核将4G空间划分成了两个部分,用户控件(0G~3G)及内核空间(3G~4G)。vmlinux在加载的时候其实并不位于3G~4G空间之上。
原因包括多个:首先PC机并未启动分页机制,它能访问的是物理内存,而无力内存有没有4G很难说。第二及时真的有4G内存,在进入保护模式之前,16位的程序也没有办法访问到这些物理内存。

因此Linux采用了一个很精巧的办法,将vmlinux加载到低端内存。再启动分页机制后将它映射到高端内存上。这对C程序来说绝对是个挑战,因为从链接的时刻各个符号的地址已经确定了,如果要写两个模式下运行的程序,每看到一个符号叫要想想现在处于什么模式,究竟是以物理内存还是用虚拟内存的方式访问这个符号。如果是以物理内存的方式访问,还要进行指针运行而后在进行类型转换,想想都头大。

简单的做法是写一段汇编代码,这段代码之后就进入分页模式,C程序向怎么写就怎么写;再次之前就要讲过一些运算了,不过这段代码比较短,整体还是可以接受的。由于进入内核之前必须先执行这段代码,因此这段代码的名字就叫做head.S(完整路径是arch/i386/kernel/head.S)。我们的代码做了很大简化,感兴趣的同学可以对比,具体代码如下:
\begin{lstlisting}

.text

#include <asm/cache.h>
#include <asm/linkage.h>
#include <asm/page.h>
	
ENTRY(startup_32)
	cld
	lgdt	boot_gdt_descr - __PAGE_OFFSET
	movl	$(__BOOT_DS), %eax
	movl	%eax, %ds
	movl	%eax, %es
	movl	%eax, %fs
	movl	%eax, %gsls

	movl %cr0,%eax
	orl $0x80000000,%eax
	movl %eax,%cr0          /* ..and set paging (PG) bit */
	ljmp $__BOOT_CS,$1f
1:
	/* Set up the stack pointer */
	lss stack_start,%esp

	/*
	 * BSS section
	 */
	.section ".bss.page_aligned","w"
ENTRY(swapper_pg_dir)
        .fill 1024,4,0
ENTRY(empty_zero_page)
	.fill 4096,1,0

/*
 * Real beginning of normal "text" segment
 */
ENTRY(stext)
ENTRY(_stext)
	
	
	.data
ENTRY(stack_start)
        .long init_thread_union+THREAD_SIZE
	.long __BOOT_DS

ready:	.byte 0

/*
* The IDT and GDT 'descriptors' are a strange 48-bit object
* only used by the lidt and lgdt instructions. They are not
* like usual segment descriptors - they consist of a 16-bit
* segment size, and 32-bit linear address value:
*/

	.globl boot_gdt_descr
	.globl idt_descr
	.globl cpu_gdt_descr
        ALIGN
# early boot GDT descriptor (must use 1:1 address mapping)
        .word 0            # 32 bit align gdt_desc.address
boot_gdt_descr:
	.word __BOOT_DS+7
	.long boot_gdt_table - __PAGE_OFFSET
	
	.word 0               # 32-bit align idt_desc.address
idt_descr:
	.word IDT_ENTRIES*8-1 # idt contains 256 entries
	.long idt_table
	
	# boot GDT descriptor (later on used by CPU#0):
	.word 0               # 32 bit align gdt_desc.address
cpu_gdt_descr:
	.word GDT_ENTRIES*8-1
	.long cpu_gdt_table
	
	.fill NR_CPUS-1,8,0   # space for the other GDT descriptors
	

	/*
	 * The boot_gdt_table must mirror the equivalent in setup.S and is
	 * used only for booting.
	 */
        .align L1_CACHE_BYTES
ENTRY(boot_gdt_table)
        .fill GDT_ENTRY_BOOT_CS,8,0
        .quad 0x00cf9a000000ffff   /* kernel 4GB code at 0x00000000 */
        .quad 0x00cf92000000ffff   /* kernel 4GB data at 0x00000000 */

	/*
	 * The Global Descriptor Table contains 28 quadwords, per-CPU.
	 */
        .align PAGE_SIZE_asm
ENTRY(cpu_gdt_table)
	.quad 0x0000000000000000   /* NULL descriptor */
	.quad 0x0000000000000000   /* 0x0b reserved */
	.quad 0x0000000000000000   /* 0x13 reserved */
	.quad 0x0000000000000000   /* 0x1b reserved */
	.quad 0x0000000000000000   /* 0x20 unused */
	.quad 0x0000000000000000   /* 0x28 unused */
	.quad 0x0000000000000000   /* 0x33 TLS entry 1 */
	.quad 0x0000000000000000   /* 0x3b TLS entry 2 */
	.quad 0x0000000000000000   /* 0x43 TLS entry 3 */
	.quad 0x0000000000000000   /* 0x4b reserved */
	.quad 0x0000000000000000   /* 0x53 reserved */
	.quad 0x0000000000000000   /* 0x5b reserved */
	
	.quad 0x00cf9a000000ffff   /* 0x60 kernel 4GB code at 0x00000000 */
	.quad 0x00cf92000000ffff   /* 0x68 kernel 4GB data at 0x00000000 */
	.quad 0x00cffa000000ffff   /* 0x73 user 4GB code at 0x00000000 */
	.quad 0x00cff2000000ffff   /* 0x7b user 4GB data at 0x00000000 */
	
	.quad 0x0000000000000000   /* 0x80 TSS descriptor */
	.quad 0x0000000000000000   /* 0x88 LDT descriptor */
	
	/* Segments used for calling PnP BIOS */
	.quad 0x00c09a0000000000   /* 0x90 32-bit code */
	.quad 0x00809a0000000000   /* 0x98 16-bit code */
	.quad 0x0080920000000000   /* 0xa0 16-bit data */
	.quad 0x0080920000000000   /* 0xa8 16-bit data */
	.quad 0x0080920000000000   /* 0xb0 16-bit data */
	/*
	* The APM segments have byte granularity and their bases
	* and limits are set at run time.
	*/
	.quad 0x00409a0000000000   /* 0xb8 APM CS    code */
	.quad 0x00009a0000000000   /* 0xc0 APM CS 16 code (16 bit) */
	.quad 0x0040920000000000   /* 0xc8 APM DS    data */
	
	.quad 0x0000000000000000   /* 0xd0 - unused */
	.quad 0x0000000000000000   /* 0xd8 - unused */
	.quad 0x0000000000000000   /* 0xe0 - unused */
	.quad 0x0000000000000000   /* 0xe8 - unused */
	.quad 0x0000000000000000   /* 0xf0 - unused */
	.quad 0x0000000000000000   /* 0xf8 - GDT entry 31: double-fault TSS */
	
\end{lstlisting}
这部代码就是Linux进入保护模式的代码,从当中可以看到Linux包含两部分的

当中还是包含很多将来要用的东西的,比如cpu\_gdt\_table。 在系统初始化阶段Linux采用boot\_gdt\_table。在正式启动之后将会切换到cpu\_gdt\_table上来。

好了到现在为止我们的系统已经进入了保护模式,接下来我们逐个初始化操作系统的子系统。

\section{构建新的内核}
这个程序虽然并不复杂,但它的构建却比一般的程序要复杂的多。因为它内部包含两种模式的代码。16位实模式代码和32位保护模式的代码,如果仅仅生成一个binary文件是非常容易的。但是要生成一个ELF文件就不那么直接了。我们必须寄出链接利器:ldscript。

在默认情况下链接器认为我们只会生成一种arch的代码,这时候只要按照默认的ldscript生成就好了。但是当涉及到两种的架构的时候,我们只能通过ldscript来指定不同架构的代码应该放在那里了。

\subsection{ldscript}
ldscript非常的容易

\subsection{模仿Linux的机制}
Linux调用lds并不像我们介绍的那么简单直接,因为它真的做了一些使的代码更容易维护的事情。但为了提高维护性,我们必须提高我们对工具的使用能力才行。
首先来介绍一下linux使用lds的方式,它并不直接使用lds文件,而是使用一种叫做lds.S的文件之后来生成lds文件,为什么这么作呢,因为linux希望所有的变量都统一的放在一个地方,而不是分散在各地。但是lds是不支持包含C语言的头文件的。但是.S文件支持,即GAS格式的汇编语言文件。

GAS格式的汇编文件有两种.s和.S,这两种文件有一个微小的不同,虽然微小但足够让你崩溃一天。.s和.S的文件都支持gcc或者as进行汇编,但是.s的文件是直接忽略include指令的,是的直接忽略还不报错,让你误以为引入了都文件但实际上没有。如果你在你的汇编程序中用到了某个\#define定义的常量,它也不报语法错误,而是报undefined symbol。这个错误如此具有误导性,以至于我会怀疑C语言通过\#define定义的常量会包含在symbol table当中。但实际上完全不会,它仅仅是没有包含头文件而已,把扩展名换成.S立时所有的问题都没有了。

