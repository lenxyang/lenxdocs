\section{Makefile基础}
Makefile最基本的单位是目标,最基本的关系是依赖。一个项目可以有多个目标组成,其中每一个目标又依赖于很多个.o文件。每一个.o文件依赖于.c文件。根据这些依赖关系可以支撑一个森林(多棵树)。如果希望生成其中任何一个目标只要从树顶开始进行后序编译即可。
\subsection{Makefile基础}
接下来我们从一个简单的设计多文件的程序来更具体的理解Makefile的工作方式。
\begin{lstlisting}[language=make]
CFLAGS = -c
default: hello
hello : hello.o lib.o
   ld  ${LDFLAGS} hello.o lib.o -o hello

hello.o: hello.c
   gcc ${CFLAGS} hello.c -o hello.o

lib.o: lib.c
   gcc ${CFLAGS} lib.c -o lib.o
\end{lstlisting}

正如之前提到的,这段代码可以用森林表示,但这森林很小仅仅有一个小树:


\qtreecentertrue 
\Tree [.default [.hello [.hello.o \textit{gcc hello.c -o hello.o} ] 
                        [.lib.o \textit{gcc lib.c -o lib.o} ] ] ]

Makefile的内部实现基本上也是以树为组织形式,而后通过后续遍历。
按照先生成依赖项而后生成自身的顺序完成整个工程的构建。

此外,make还有一个很好的特性:它会监控每一个依赖项及target的变化。如果其中任何一个改变,那么所有依赖于它的项(包括直接依赖及间接依赖)都会被重新生成。这个特性很好,尤其是当项目设计非常多的文件的时候,如果一个文件变化导致整个项目重新build那就太耗时了。
\subsection{伪目标}
实际上Makefile会将所有的target及依赖项当作文件,如果目标不是一个文件会怎么样呢?make会把它当作文件并理所当然的认为它被存在是因为没有被生成,因此这个目标总是被执行。最常见的伪目标是clean。例如给上面的Makefile加上clean:
\begin{lstlisting}[language=make]
CFLAGS = -c
default: hello
hello : hello.o lib.o
   ld  ${LDFLAGS} hello.o lib.o -o hello

hello.o: hello.c
   gcc ${CFLAGS} hello.c -o hello.o

lib.o: lib.c
   gcc ${CFLAGS} lib.c -o lib.o

clean:
    rm -f *.o
    rm -f hello
.PHONY: clean
\end{lstlisting}

每当我们运行make clean的时候,两条rm命令都会执行,因为clean是不存的。make提供了一个特殊的指令叫做.PHONY,它可以用于表示伪目标,但一般不会这么做。

Makefile的内部实现基本上也是以树为组织形式,而后通过后续遍历。
按照先生成依赖项而后生成自身的顺序完成整个工程的构建。

此外,make还有一个很好的特性:它会监控每一个依赖项及target的变化。如果其中任何一个改变,那么所有依赖于它的项(包括直接依赖及间接依赖)都会被重新生成。这个特性很好,尤其是当项目设计非常多的文件的时候,如果一个文件变化导致整个项目重新build那就太耗时了。

\subsubsection{强制重新生成}
在这里顺便给大家介绍一个Linux makefile的小技巧,在kbuild Makefiles当中,你可能经常会发现一个依赖项FORCE, 它其实是用来强制重新生成的。如果加上这个依赖项无论其他依赖上如何,目标都将被重新生成。例如在顶层Makefile当中。
\begin{lstlisting}[language=make]
include/linux/version.h: $(srctree)/Makefile FORCE
        $(call filechk,version.h)
\end{lstlisting}
无论何时make执行到它都一定会执行,那么FORCE是如何实现的呢,简单到难以置信,它是一个伪目标。
\begin{lstlisting}[language=make]
FORCE:

\end{lstlisting}

\subsection{Makefile变量}
和众多脚本语言一样,Makefile怎么可能没有变量呢。上面的例子就有两个变量\$\{LDFLAGS\}和变量\$\{CFLAGS\}。
变量定义的格式比较直接一个赋值运算符就可以搞定了。


\subsubsection{赋值运算}
GNU Make当中包含两种复制运算符“=”和“:=”。这两个运算符还是有一些差异的。赋值符号“:=“会立刻计算右边的字符串并赋值给变量。而运算符“=”则会仅仅保存表达式,在用到变量的一刻,make程序会计算(evaluate)它的值并加入到运算当中。我们通过一个例子来了解一下他们的不同。这非常重要,因为Linux的Makefile当中这种技巧用了很多。我们还是用个简单的例子给大家一个直观的感受。这个例子设计多个Makefile。

\begin{lstlisting}[language=make]
cur  =  target1
v1   = ``current: ${cur}''
v2  := ``current: ${cur}''

cur=target2
v3 = ${v1}
v4 = ${v2}

default: ${target1} ${target2}
        @echo  ${v3}
        @echo  ${v4}
\end{lstlisting}

他的结果是
\begin{lstlisting}[language=bash]
current: target2
current: target1
\end{lstlisting}
从结果可以看到在cur的值改变以后v1的值改变了,但是v2的值没有变。

\subsubsection{导出变量}
在涉及到多个Makefile协同工作的时候,经常要在多个Makefile之间共享变量。导出变量是共享变量的常用方法,这种方法尤其是用于导出常数,比如LDFLAGS, CFLAGS,CFLAG\_KERNEL这样的通用编译选项,都是用这种方法的,具体格式为:
\begin{lstlisting}[language=make]
CFLAGS  =  ....
LDFLAGS =  ....
export CFLAGS LDFLAGS CFLAGS_KERNEL ...
\end{lstlisting}

\subsection{Automatic Variables}
大家注意到在每条规则下面都写了对应的生成指令,其实还有更好的方法。
make提供一些特殊变量,Automatic Variables是其中一种,下面列出几个我们用到的变量,如果读者想进一步了解可以参考GNU Make的手册。
\begin{compactenum}
\item \$@    当前规则的目标文件名
\item \$<    当前对则的第一个依赖项
\item \$?    当前对则的所有依赖项
\item \$(@D) 当前规则文件名当中的目录部分
\end{compactenum}
知道这几个特殊变灵之后相信大家已经知道如何改造Makefile了,例如ld命令可以改写成。
\begin{lstlisting}[language=make]
ld ${ldflags} $? -o $@
\end{lstlisting}
使用类似的方法,我们可以将编译命令改写成这样。
\begin{lstlisting}[language=make]
gcc ${cflags} -c $< -o $@
\end{lstlisting}

\subsection{隐含规则}
如果文件多了,每个文件都要写一遍还是很麻烦的。好在make还提供了“隐含规则”(Implicit Rules)的功能。
\begin{lstlisting}[language=make]
CFLAGS = -c

%.o : %c
  gcc ${CFLAGS} %< -o $@

default: hello
hello : hello.o lib.o
   ld  ${LDFLAGS} hello.o lib.o -o hello

hello.o: hello.c
lib.o: lib.c

clean:
    rm -f *.o
    rm -f hello
.PHONY: clean
\end{lstlisting}

\subsection{Makefile函数调用}
如果单单认为Makefile就是这么简单的话,恐怕会让大家失望。
如果只是这么简单,我们也就不用单独开辟一章来介绍Linux Kernel的Makefile了。
实际上Makefile包含非常多的强大功能,对于很多支持多平台的软件来说,这些功能都是不可或缺的。
接下来我们介绍函数调用。
\subsection{Makefile流程控制}
大部分介绍编程语言书籍来都倾向于将流程控制放在函数之前,这这里我们却不得不成为一个例外。
原因很简单,Makefile的流程控制是基于函数调用的,它并没有关键字if,或是关键字for,
它拥有的是两个函数if和foreach。
这两个函数是Makefile流程控制的基础。
\subsubsection{if}
\subsubsection{foreach}
还是通过例子来学习它吧
\begin{lstlisting}[language=make]
srcs := a.c b.c c.c
files := $(foreach,n,${srcs},$(shell gcc -c ${n} -o ${n}.o))
\end{lstlisting}
函数for有4个参数
\begin{itemize}
\item n 用于枚举,代表一次迭代从参数srcs当中取出的值
\item \$\{srcs\} 用于枚举的对象
\item 循环体 \$(shell gcc \-c \$\{n\} \-o \$\{n\}.o) 
\end{itemize}

\subsection{多Makefile协同工作}
对于大型的项目来说多个Makefile是一定的,多Makefile的相同方式也有多种,比较成用的是直接通过make调用,或者include的方式引入。
\subsubsection{include}
这种方式的一般做法是 "从Makefile"仅仅定义target,"主Makefile"则将它们组织起来。著名开源软件chromium就采用这种方式,表面看起来chromium采用的gyp,它其实是一个生成Makefile的脚本,最终完成build的工作的还是Makefile。在这里我们稍微加少一下gyp的工作方式。假设项目现在分成了三个部分liba, libb和程序c。每一个小项目都有一个gyp的文件,这个文件会为各个项目的文件生成一个Makefile。假定liba包含一个名为a.c的文件,libb包含一个名为b.c的文件,程序c包含一个c.c的文件。那么它会生成4个Makefile。

首先是根目录下的Makefile,它包含了所有相关的Makefile,并定义了最顶层的目标,最顶层依赖的目标则放在include的Makefile当中。
\begin{lstlisting}[language=make]
include liba/liba.mk
include libb/libb.mk
include c/c.mk
default: ${target}/c
\end{lstlisting}

\begin{lstlisting}[language=make]
${target}/c/c: ${target}/liba/liba.a ${target}/libb/libb.a
\end{lstlisting}

接下来是两个库文件的Makefile。
\begin{lstlisting}[language=make]
${target}/liba/liba.a : liba/a.o
\end{lstlisting}

\begin{lstlisting}[language=make]
${target}/libb/libb.a : libb/b.o
\end{lstlisting}
gyp会根据依赖关系生成include,make程序会根据所有相关的Makefile(根Makefile和其他被引入的Makefile)生成一个依赖树。其实include的本质就是让共享target及target生成规则。

另外一点大家可能注意到了,所有的target都使用了全局路径。对于chomium这种规模的项目来说,重名是不可避免的,于是所有的生成文件及依赖文件,gyp都使用相对于项目根目录的完整路径表达,这样就没有问题了。据说google内部使用的blade也采用类似的方式。不同的是gyp支持多平台,使用json作为数据格式,而google blade则直接使用python程序来管理。腾讯在google code上开源了一个与google blade类似的工具,它家可以看看。

这种方式的Makefile写起来是很麻烦的,而且大部分内容都是重复的,因此使用这种方式管理项目时,一般都会使用另外一个工具来生成Makefile,这些用来生成Makefile的工具的语法都很简单,可读性也非常的好。贴一段blade的语法吧

\begin{lstlisting}[language=make]
cc_binary(name = 'exec',
          srcs = ['a.c',
                  'b.c',
                  ],
          deps = ['//base:string',
                 ]
)
\end{lstlisting}
稍微解释一下,这个项目的类型是一个可执行程序,名为exec,包括两个源文件a.c和b.c,项目依赖于一个名为//base:string的库(当中的路径还告诉的blade如何找到以来的BUILD文件),看起来要比Makefile的语法舒服多了吧。

\subsubsection{make -C }
这是Linux采用的方式,在顶层Makefile当中通过make命令调用子目录下的Makefile。与include的方式不同,这种方式调用Makefile一般是通过变量来传递数据的。

以Linux为例,它通过obj-y定义了所有用来生成vmlinux的文件,在顶层Makefile当中,会将所有的项逐个生成输出文件并将它们链接起来。


\subsection{其他技巧}
\subsubsection{--dry-run}
make提供了一个参数--dry-run,顾名思义它仅仅打印步骤去并执行任何的生成目标的指令。用它来观察Makefile的行为非常好用,尤其是在阅读复杂的Makefile的时候。
