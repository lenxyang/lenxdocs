\section{Makefile基础}
Makefile最基本的单位是目标,最基本的关系是依赖。一个项目可以有多个目标组成,其中每一个目标又依赖于很多个.o文件。每一个.o文件依赖于.c文件。根据这些依赖关系可以支撑一个森林(多棵树)。如果希望生成其中任何一个目标只要从树顶开始进行后序编译即可。
\subsection{Makefile基础}
接下来我们从一个简单的设计多文件的程序来更具体的理解Makefile的工作方式。
\begin{lstlisting}[language=make]
CFLAGS = -c
default: hello
hello : hello.o lib.o
   ld  ${LDFLAGS} hello.o lib.o -o hello

hello.o: hello.c
   gcc ${CFLAGS} hello.c -o hello.o

lib.o: lib.c
   gcc ${CFLAGS} lib.c -o lib.o
\end{lstlisting}

正如之前提到的,这段代码可以用森林表示,但这森林很小仅仅有一个小树:


\qtreecentertrue 
\Tree [.default [.hello [.hello.o \textit{gcc hello.c -o hello.o} ] 
                        [.lib.o \textit{gcc lib.c -o lib.o} ] ] ]


Makefile的内部实现基本上也是以树为组织形式,而后通过后续遍历。
按照先生成依赖项而后生成自身的顺序完成整个工程的构建。


此外,make还有一个很好的特性:它会监控每一个依赖项及target的变化。如果其中任何一个改变,那么所有依赖于它的项(包括直接依赖及间接依赖)都会被重新生成。这个特性很好,尤其是当项目设计非常多的文件的时候,如果一个文件变化导致整个项目重新build那就太耗时了。
\subsection{伪目标}
实际上Makefile会将所有的target及依赖项当作文件,如果目标不是一个文件会怎么样呢?make会把它当作文件并理所当然的认为它被存在是因为没有被生成,因此这个目标总是被执行。最常见的伪目标是clean。例如给上面的Makefile加上clean:
\begin{lstlisting}[language=make]
CFLAGS = -c
default: hello
hello : hello.o lib.o
   ld  ${LDFLAGS} hello.o lib.o -o hello

hello.o: hello.c
   gcc ${CFLAGS} hello.c -o hello.o

lib.o: lib.c
   gcc ${CFLAGS} lib.c -o lib.o

clean:
    rm -f *.o
    rm -f hello
.PHONY: clean
\end{lstlisting}
每当我们运行make clean的时候,两条rm命令都会执行,因为clean是不存的。make提供了一个特殊的指令叫做.PHONY,它可以用于表示伪目标,但一般不会这么做。

Makefile的内部实现基本上也是以树为组织形式,而后通过后续遍历。
按照先生成依赖项而后生成自身的顺序完成整个工程的构建。

此外,make还有一个很好的特性:它会监控每一个依赖项及target的变化。如果其中任何一个改变,那么所有依赖于它的项(包括直接依赖及间接依赖)都会被重新生成。这个特性很好,尤其是当项目设计非常多的文件的时候,如果一个文件变化导致整个项目重新build那就太耗时了。
\subsection{Makefile变量}
和众多脚本语言一样,Makefile怎么可能没有变量呢。上面的例子就有两个变量\$\{LDFLAGS\}和变量\$\{CFLAGS\}。
变量定义的格式比较直接一个赋值运算符就可以搞定了。

GNU Make当中包含两种复制运算符“=”和“:=”。这两个运算符还是有一些差异的。赋值符号“:=“会立刻计算右边的字符串并赋值给变量。而运算符“=”则会仅仅保存表达式,在用到变量的一刻,make程序会计算(evaluate)它的值并加入到运算当中。我们通过一个例子来了解一下他们的不同。这非常重要,因为Linux的Makefile当中这种技巧用了很多。

\subsection{Makefile函数调用}
如果单单认为Makefile就是这么简单的话,恐怕会让大家失望。
如果只是这么简单,我们也就不用单独开辟一章来介绍Linux Kernel的Makefile了。
实际上Makefile包含非常多的强大功能,对于很多支持多平台的软件来说,这些功能都是不可或缺的。
接下来我们介绍函数调用。
\subsection{Makefile流程控制}
大部分介绍编程语言书籍来都倾向于将流程控制放在函数之前,这这里我们却不得不成为一个例外。
原因很简单,Makefile的流程控制是基于函数调用的,它并没有关键字if,或是关键字for,
它拥有的是两个函数if和foreach。
这两个函数是Makefile流程控制的基础。
\subsubsection{if}
\subsubsection{foreach}
还是通过例子来学习它吧
\begin{lstlisting}[language=make]
srcs := a.c b.c c.c
files := $(foreach,n,${srcs},$(shell gcc -c ${n} -o ${n}.o))
\end{lstlisting}
函数for有4个参数
\begin{itemize}
\item n 用于枚举,代表一次迭代从参数srcs当中取出的值
\item \$\{srcs\} 用于枚举的对象
\item 循环体 \$(shell gcc \-c \$\{n\} \-o \$\{n\}.o) 
\end{itemize}
