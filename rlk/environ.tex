\chapter{工作环境}

\section{qemu}
\subsection{为什么选用qemu}
比较流行的用于调试内核的虚拟机有很多,Oracle VirtualBox, VMWare, qemu, bochs等等。
前面两款是著名的商业软件,我们暂不考虑。 
作者选用qemu作为虚拟机的原因主要有以下几个
\begin{compactenum}
  \item 内嵌gdbstub调试 这可能是使用它的最大原因了,从虚拟机启动开始就可以直接使用gdb远程调试了,没有比这更然人舒心了。
bochs也包含gdbstub,但需要用户自己编译,如果
  \item 支持多种架构
  \item 速度很快
  \item 安装容易,配置简单,速度快,体积小
\end{compactenum}

\section{bochs}
作为虚拟机bochs的速度真的很慢,但是他内嵌了调试功能,这个对于刚开始接触汇编程序和x86保护模式编程的人来说,真的是不可多得。

直接使用它的内嵌调试器可以查看gdt、idt、 ldt、堆栈等等。这个功能用起来真的非常方便。

举个作者亲身的经历吧,在刚开始学习保护模式编程的时候,作者遇到了一个怪事。
程序倒是很简单,初始化idt, 然后使用ud2触发异常,而后通过汇编程序调用C函数将错误信息打印在屏幕上。
因为涉及到asm调用C程序,作者使用堆栈传递参数,但是push指令却总不起作用。
一怒之下改用寄存器传递参数,发现疑点问题没有。这个问题一定出现在了堆栈上。
于是作者用bochs的dump-stack命令查看堆栈,发现堆栈的地址是16位的,没错它仅仅使用了esp的低16位地址。
回溯代码,作者终于发现原来作者创建的堆栈段是16位的,而我用gdb调试时都是通过观察esp的地址来查看的,所以一直没法查处问题所在。
使用bochs调试器时,它可没有做的先入为主的错误观念,于是很容易发现问题所在。
bochs存在的意义就是帮助我们double-check。
尤其是当你认为是虚拟机bug的时候,多一个虚拟机用来验证总是一件好事。

\subsection{bochs gdbstub}

bochs最大的缺陷就在于没有内嵌gdbstub, 需要用户自己把它编译进去。这也是一个充满各种小坑的路。
作者开始编译的时候使用cygwin配置(configure), 使用VC8编译,编译时成功了,但是一旦gdb成功连接,bochs立马崩溃。
还好后来在cygwin下使用mingw编译成功,如果读者想用,我可以直接把编译成功的binary上传,下载之后应该可以直接使用。

\subsection{cygwin下编译bochs with gdbstub}
前面提到bochs提供的binary是不支持gdbstub的,如果希望编译按照这个说明来实现。它依赖于一些常用工具,我们务必在运行configure脚本之前将它们装好。这些工具的列表在这,相信用过cygwin的朋友们都会安装。

\begin{lstlisting}[language=bash]
CC="/usr/bin/i686-pc-mingw32-gcc"
CXX="/usr/bin/i686-pc-mingw32-g++"
CFLAGS="-static-libstdc++ -static-libgcc -O3"
CXXFLAGS=$CFLAGS

export CC
export CXX
export CFLAGS
export CXXFLAGS

./configure  --enable-ne2000 \
    --enable-all-optimizations \
    --enable-cpu-level=6 \
    --enable-x86-64 \
    --enable-pci \
    --enable-clgd54xx \
    --enable-usb \
    --enable-usb-ohci \
    --enable-e1000 \
    --enable-show-ips \
    --enable-gdb-stub \
    --with-win32 --with-rfb --with-nogui
# --enable-sb16
\end{lstlisting}
\section{操作系统}
从开始接触Linux,作者的工作一般都是使用笔记本连接远程开发机的。
这导致一直都不适应linux的桌面环境,如果非用不可的话,作者宁愿安装一个虚拟机然后用putty链接,就好象链接远程机器一样。
在刚开始学习内核编程的时候,作者也一直采用这种方式。
自从cygwin的终端使用putty实现之后,我就毫不犹豫的投入到cygwin的怀抱。
用cygwin就像putty链接虚拟机一样,此外还省去了共享文件,或者远程拷贝文件的麻烦。
cygwin也有它的缺点,xserv连接器了比较麻烦,不过还好我们需要用到的需要x的软件都有windows版。
cygwin的另外一个严重缺点是它之上的编译工具,GNU等生成的代码都是PE文件格式的,不过我们可以通过交叉编译来解决这个问题。
\section{开发工具}

上面提过cygwin下集成的gcc,mingw等开发工具生成的程序都是可以直接在windows下运行的。
因为它产生的是PE文件,一种windows采用的文件格式。
但是我们的内核可是linux的,为此我们需要生成ELF格式的可支持文件,没办法我们必须生成一组交叉编译工具。

如果读者使用Linux作为桌面开发环境有一点也需要注意,本书的内容全部基于i386开发,即:32位操作系统。
但目前流行的Linux已经全部是64位了,所以大家在编译的时候也需要指定arch位i386啊。

\section{第一个程序}

万事俱备,我们用一个程序来启动机器吧。
\begin{lstlisting}
	.code16
	.text
_start:	
	movw	%cs, %ax
	movw	%ax, %ds

	movl	$0x7FF0, %esp
	xor	%ebp, %ebp
1:
        jmp     1b;

__puts:
	pusha
	xor	%bh, %bh
.Lputs1:
	lodsb
	test	%al, %al
	jz	.done
	movb	$0x0e, %ah
	int	$0x10
	jmp	.Lputs1
.done:	
	popa
	ret
msg:
	.asciz	"Hello, Kernel...\r\n"
.org 510
	.word	0xAA55
\end{lstlisting}
\subsection{编译运行}
