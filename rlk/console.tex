\chapter{控制台及终端}

\section{printk}
\subsection{early-printk}
early-printk是初始化时期的printk的实现。Linux为了提供类似功能的通用初始化架构,利用了一些技巧,这些技巧很多其他模块都用到了。


setup\_early\_printk是初始化early\_printk的函数,通过宏early\_param将初始化printk的函数setup\_early\_printk的存放在obs\_kernel\_param数组当中。

\begin{lstlisting}[language=c]

// 文件 arch/x86/kernel/early_printk.c
early_param("earlyprintk", setup_early_printk);


// 文件 include/kernel/init.h
/*
 * Only for really core code.  See moduleparam.h for the normal way.
 *
 * Force the alignment so the compiler doesn't space elements of the
 * obs_kernel_param ``array'' too far apart in .init.setup.
 */
#define __setup_param(str, unique_id, fn, early)                        \
        static const char __setup_str_##unique_id[] __initconst \
                __aligned(1) = str; \
        static struct obs_kernel_param __setup_##unique_id      \
                __used __section(.init.setup)                   \
                __attribute__((aligned((sizeof(long)))))        \
                = { __setup_str_##unique_id, fn, early }

#define __setup(str, fn)                                        \
        __setup_param(str, fn, fn, 0)

/* NOTE: fn is as per module_param, not __setup!  Emits warning if fn
 * returns non-zero. */
#define early_param(str, fn)                                    \
        __setup_param(str, fn, fn, 1)

\end{lstlisting}
从上面的代码可以看到,\_\_setup\_param的数据以struct obs\_kernel\_param的结构存放并置与.init.setup段当中,这个段专门用来存储初始化阶段的数据,且全部以结构体obs\_kernel\_param的结构存储。通过段指令,即使early\_param宏存放与不同的文件当中,但数据仍然是连续存储的,它看起来就像一个数组。

连续存放的文件解决了,接下来探究一下Linux是如何访问这些数据的,毕竟每次修改代码,这些数据的位置都会改变,因此它们的地址是不固定的,也就是说不能通过制定地址常量的方式来访问。Linux通过定义在lds当中的全局变量来定位.init.setup的起始及结束地址,这种技巧也是Linux非常常用的技巧之一。
\begin{lstlisting}
__setup_start = .;
.init.setup : { *(.init.setup) }
__setup_end = .;
\end{lstlisting}

而后在init/main.c当中就可以通过两个外部变量来遍历这个数组了。
\begin{lstlisting}[language=c]

extern struct obs_kernel_param __setup_start[], __setup_end[];
static int do_early_param(char* param, char* val) {
  for (p = __setup_start; p < __setup_end; p++) {
    /* code */
  }
}
\end{lstlisting}
