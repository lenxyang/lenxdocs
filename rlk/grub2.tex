\chapter{使用grub引导}
分析Linux代码是必不可少的,本章将介绍如何在磁盘镜像上安装grub并使用grub引导Linux。
\section{安装GRUB2}
\subsection{创建磁盘镜像}
创建磁盘的方法很多,本质上只是创建一个指定大小的内容全部为0的文件。这里采用bximage创建磁盘,它会顺便计算出磁盘的磁道,扇区数等信息,在创建分区的时候这些信息都是必要的。

\begin{lstlisting}[language=bash]
# 创建一个500M的磁盘
 bximage
 Do you want to create a floppy disk image or a hard disk image?
 Please type hd or fd. [hd] 

 What kind of image should I create?
 Please type flat, sparse or growing. [flat]

 Enter the hard disk size in megabytes, between 1 and 129023
 [10] 500

 I will create a 'flat' hard disk image with
  cyl=1015
  heads=16
  sectors per track=63
  total sectors=1023120
  total size=499.57 megabytes
\end{lstlisting}


在完成磁盘的创建以后,还需要对磁盘创建分区并格式化。Linux有一个非常方便的工具mtools可以完成功能,它提供了一组命令不需要将磁盘镜像映射成设备就可以对其操作。默认情况下linux是包含这个工具包的,可以通过命令sudo apt-get install mtools来安装。
\begin{lstlisting}[language=bash]
# mpartition -I C:
$ mpartition -cpv -t 1015 -h 16 -s 63 -b 63 C:
$ mformat C:
\end{lstlisting}

\subsection{加载磁盘}
在完成磁盘的创建以后,需要对磁盘镜像以块设备的方式进行操作。
\begin{lstlisting}[language=bash]
$ sudo losetup /dev/loop0 c.img
$ ls -l /dev/loop0
brw-rw---- 1 root disk 7, 0 Feb  4 14:56 /dev/loop0 # 7, 0 是主,次设备号
# 为磁盘的分区建立设备映射
$ sudo kpartx -v -a /dev/loop0
# 将分区映射到设备 /dev/loop1上
$ sudo losetup /dev/loop1 /dev/mapper/loop0p1
# 格式化磁盘
\end{lstlisting}

\subsection{安装grub2}

\begin{lstlisting}[language=bash]
$ sudo grub-install --no-floppy \
       --root-directory=/mnt \
       --modules=''fat ext2 part_msdos''  /dev/loop1
\end{lstlisting}

将c.img作为启动盘启动qemu,此时已经可以看到GRUB2的提示符号了
\begin{lstlisting}
qemu-system-i386 -hda c.img
\end{lstlisting}

\subsection{关闭磁盘映射}
\begin{lstlisting}[language=bash]
# 关闭映射设备
$ sudo kpartx -d /dev/loop1
$ sudo losetup -d /dev/loop1
$ sudo losetup -d /dev/loop0
\end{lstlisting}
\section{安装Linux}
\subsection{安装内核文件}
将编译成功的Linux拷贝到文件镜像下,通常这个文件处于
arch/x86/boot/vmlinux
\begin{lstlisting}[language=bash]
mcopy arch/x86/boot/vmlinux c:
\end{lstlisting}
\subsection{配置grub}
打开一个文件并输入
\begin{lstlisting}[language=bash]

menuentry "My Linux Kernel" {
  set root='(hd0,msdos1)'
  linux /linux/vmlinuz ro
  initrd /linux/initrd.img
  insmod gzio
  insmod part_msdos
  insmod ext2
}
mcopy grub.cfg c:/boot/
\end{lstlisting}


\subsection{启动}
启动qemu,在启动grub cli之后输入
\begin{lstlisting}[language=bash]
grub> configfile /boot/grub.cfg
grub> boot
\end{lstlisting}
此时应该可以在屏幕当中看到内核启动了。
\section{安装GRUB2}
