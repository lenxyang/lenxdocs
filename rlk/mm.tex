
\chapter{Linux内存管理}

内存管理是内核当中最复杂的部分了,而且它是很多模块的基础,在没有内存管理之前,操作内存的基本方式是“按照约定”:指定的物理内存存放制定的内容。这样做的好处是简单,但无法完成动态分配等操作。在使用C语言时,很多时候都用全局变量和静态变量来管理内存。

Linux支持多种体系结构,为了方便这种跨体系结构的开发, Linux使用了大量的类似于C++ Virtual Function的方式。
首先它为每一个子系统定义了一组通用操作,在不同的系统下有不同的实现。
Linux在编译期间可以根据宏定义来include对应的头文件编译对应的c文件。
MM(memory management)与Linux其他模块一样也是按照这种方式组织的, 首先在根目录mm下定义了通用的操作及通用的逻辑实现。
与平台相关代码则放在了arch/i386/mm下, 这其中包括i386 mm的初始化及差异化的功能实现。

我们从i386的内存分页机制及初始化流程来开始,一步一步分析,一步一步实现。
在基础打好之后,我们在研究更高层的linux内存管理策略。

